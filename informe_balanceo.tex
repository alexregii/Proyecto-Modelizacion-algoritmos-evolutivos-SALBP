\documentclass[12pt]{report}

\title{Proyecto de modelización - Programación evolutiva\\ \textbf{El problema de balanceo de líneas de montaje}} 
\author{Paul Rohel - Alejandro Regidor Orellana}
\date{\today}

\usepackage[spanish]{babel}
\usepackage[utf8]{inputenc}
\usepackage[T1]{fontenc}
\usepackage{graphicx}
\usepackage{mathtools, amsfonts, amssymb, amsthm}
\usepackage{float}

\renewcommand{\chaptername}{}

\setlength{\parindent}{0pt}

\makeatletter
\renewcommand{\@makechapterhead}[1]{\vspace*{50\p@}%
  \begin{center}%
    {\normalfont\huge\bfseries #1\par}%
  \end{center}\vskip 40\p@}
\makeatother

\begin{document}

\maketitle
\tableofcontents

\chapter{Planteamiento del problema}

El problema de balanceo de líneas de ensamblaje puede plantearse de diferentes maneras.Dadas $n$ tareas, cada una con un tiempo de procesamiento asignado y una lista de relaciones de precedencia entre ellas, dichas tareas deben asignarse a $m \leq n$ estaciones de trabajo.\\
Una de las variantes clásicas, conocida como \textit{Simple Assembly Line Balancing Problem tipo 1} (SALBP-1), consiste en encontrar una asignación de las $n$ tareas a las estaciones de manera que se minimice el número de estaciones, dado un tiempo de ciclo preestablecido.\\
En cambio, la variante sobre la que hemos trabajado (SALBP-2) tiene como objetivo encontrar una asignación de tareas que minimice el tiempo de ciclo, dado un número fijo de estaciones $m$. En este contexto, el tiempo de ciclo se define como el tiempo total más largo asignado a una estación, es decir, el máximo entre todas las estaciones.

\chapter{Presentación general del algoritmo genético}

Para resolver este problema, implementamos un algoritmo genético. 
Dadas $n$ tareas que deben asignarse a $m$ estaciones, el espacio de búsqueda contiene $O\left(\binom{n}{m}\right)$ soluciones posibles. 
Una solución, es decir, una asignación de las tareas entre las diferentes estaciones, se codifica posicionalmente como un arreglo unidimensional indexado de $0$ a $n-1$, 
cuyos elementos toman valores enteros entre $0$ y $m-1$.

Por ejemplo, el arreglo $[0, 0, 0, 1, 0, 2, 2, 1]$ representa una solución para un problema con $8$ tareas distribuidas en $3$ estaciones.\\

Para codificar las relaciones de precedencia entre las tareas, utilizamos una matriz $A$ de tamaño $n \times n$ definida de la siguiente manera:\\
$
\forall i,j \in \{0,\dots,n-1\},$\\
$
A(i,j) = 
\begin{cases}
1 & \text{si la tarea } j \text{ precede directamente a la tarea } i \\
0 & \text{en caso contrario}
\end{cases}
$\\

Dado que todas las tareas que deben preceder a otra deben asignarse a una estación anterior o, como máximo, a la misma estación que la tarea que las sigue, 
las relaciones de precedencia imponen restricciones estrictas sobre la validez de las soluciones. También es necesario verificar que exactamente todos los grupos dados, y solo esos, aparezcan en los valores del arreglo solución.\\

La función de evaluación de un individuo consiste esencialmente en devolver el tiempo más largo entre las estaciones. Penaliza significativamente a los individuos no válidos, 
asignándoles un puntaje arbitrariamente grande.
Por lo tanto, el objetivo del problema será encontrar el individuo que minimice esta función.































\end{document}